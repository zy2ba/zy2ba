% !TeX spellcheck = russian-aot
% !TXS template
\documentclass[11pt,a4paper,sans,english,russian]{moderncv}        % possible options include font size ('10pt', '11pt' and '12pt'), paper size ('a4paper', 'letterpaper', 'a5paper', 'legalpaper', 'executivepaper' and 'landscape') and font family ('sans' and 'roman')
\moderncvstyle{classic}                             % style options are 'casual' (default), 'classic', 'oldstyle' and 'banking'
\moderncvcolor{blue}                               % color options 'blue' (default), 'orange', 'green', 'red', 'purple', 'grey' and 'black'
\moderncvicons{awesome}
%\nopagenumbers{}                                  % uncomment to suppress automatic page numbering for CVs longer than one page
\usepackage[T2A]{fontenc}
\usepackage[utf8]{inputenc}                       % if you are not using xelatex ou lualatex, replace by the encoding you are using
\usepackage{lmodern}

\usepackage[scale=0.75,a4paper]{geometry}
\usepackage{babel}

\usepackage[unicode]{hyperref}
\definecolor{linkcolour}{rgb}{0,0.2,0.6}
\hypersetup{colorlinks,breaklinks,urlcolor=linkcolour, linkcolor=linkcolour}

%----------------------------------------------------------------------------------
%            personal data
%----------------------------------------------------------------------------------
\firstname{Илья}%<first name%:columnShift:-1,persistent%>
\familyname{Зяблицев}%<family name%:columnShift:-1,persistent%>
\title{Frontend-разработчик (Angular)}%<Resumé title%:columnShift:-1,persistent%>                               % optional, remove/comment the line if not wanted
\address{г~Киров}{}{}%<street and number%:columnShift:-5,persistent%>%<postcode city%:columnShift:-3,persistent%>%<country%:columnShift:-1,persistent%>         % optional, remove/comment the line if not wanted; the "country" arguments can be omitted or provided empty
\mobile{\href{tel:+79123679963}{+7~(912)~367-9963}}%<mobile number%:columnShift:-1,persistent%>                          % optional, remove/comment the line if not wanted
%\phone{}%<phone number%:columnShift:-1,persistent%>                           % optional, remove/comment the line if not wanted
%\fax{}%<fax number%:columnShift:-1,persistent%>                             % optional, remove/comment the line if not wanted
\email{zy2bas@gmail.com}%<email%:columnShift:-1,persistent%>                               % optional, remove/comment the line if not wanted
%\homepage{}%<home page%:columnShift:-1,persistent%>                         % optional, remove/comment the line if not wanted
\extrainfo{\faPaperPlane\href{https://t.me/zy2ba}{~@zy2ba}}%<additional information%:columnShift:-1,persistent%>                 % optional, remove/comment the line if not wanted
\photo[64pt][0.1pt]{photo.png}%<photo.jpg%:columnShift:-1,persistent%>                       % optional, uncomment the line if wanted; '64pt' is the height the picture must be resized to, 0.4pt is the thickness of the frame around it (put it to 0pt for no frame) and 'picture' is the name of the picture file
%\quote{}%<some quote%:columnShift:-1,persistent%>                                 % optional, remove/comment the line if not wanted
%
\renewcommand{\sfdefault}{cmr}

\usepackage{url}
\usepackage{xcolor}
\usepackage{microtype}

\def\UrlFont{\em}

\begin{document}


%-----       resume       ---------------------------------------------------------
   \makecvtitle


   \section{Образование}\label{sec:education}
      \cventry{--}{Бакалавриат, 2017~г}{Вятский государственный университет}{\newline кафедра ЭВМ}{направление "Информатика и вычислительная техника"}{}


   \section{Резюме}\label{sec:resume}
      \cvitem{--}{Опыт в Angular и Typescript -- 2+ года}
      \cvitem{--}{Опыт в JavaScript разработке -- 3 года}
      \cvitem{--}{Опыт работы с Rx фреймворками (RxJs, RxJava, RxSwift) -- 3 года}
      \cvitem{--}{Есть опыт работы с Nx monorepo, NgRx, Polymorpheus, Angular Universal, Angular Material}
      \cvitem{--}{Есть опыт опыт написания Unit-тестов с применением Jasmine и Jest}
      \cvitem{--}{Есть опыт удалённой работы в крупных компаниях по outstaffing. Работал над \hyperref[subsec:infomat]{информационными киосками ЕМИАС в Solit~Clouds}, а так же над \hyperref[subsec:kb-kr]{системой онлайн-банкинга iSimpleBank в iSimpleLab}}
      \cvitem{--}{Имею опыт работы с NodeJS фреймворками (NestJS, Koa, Express) с использованием ORM Typeorm и Sequelize}
      \cvitem{--}{\hyperref[phantomsec:gymbo-front-lead]{Есть опыт работы ведущим разработчиком по схеме "Играющий тренер".} До 3 фронтенд-разработчиков под руководством. \newline В обязанности входило: ревью кода, делегирование задач, планирование и контроль сроков, проработка и/или утверждение реализации функционала, помощь в решении задач. В свободное время брал задачи и решал лично}
      \cvitem{--}{\hyperref[phantomsec:gymbo-lead]{Есть опыт управления продуктовой командой из 8 человек.} \newline В обязанности входило: постановка задач, планирование и контроль сроков, утверждение ключевых технических решений}


   \section{Участие в open-source проектах}\label{sec:opensource}
      \cvitem{}{Имею небольшой личный вклад в open-source проекты}

      \subsection{Успешные PR}
         \cvlistitem{Taiga UI \url{https://github.com/TinkoffCreditSystems/taiga-ui/pull/348}}
         \cvlistitem{Tinkoff Linters \url{https://github.com/TinkoffCreditSystems/linters/pull/63}}
         \cvlistitem{ngx-forms-typed \url{https://github.com/gparlakov/forms-typed/pull/34}}

      \subsection{Полезные Issues}
         \cvlistitem{ng-image-slider \url{https://github.com/sanjayV/ng-image-slider/issues/108} }
         \cvlistitem{Permissions API for Angular \url{https://github.com/ng-web-apis/permissions/issues/5} }
         \cvlistitem{NgDompurify \url{https://github.com/TinkoffCreditSystems/ng-dompurify/issues/68} }


   \section{Участие в проектах}\label{sec:projects}

      \subsection{Gymbo}\label{subsec:gymbo}
         \cvitem{Период участия}{Январь 2020 -- настоящее время}
         \cvitem{Описание}{Платформа с каталогом фитнес-клубов с возможностью покупки абонементов в зал и персональных тренировок. Программный комплекс включает сайт, мобильное приложение, бэкенд и панель администратора сервиса}
         \cvitem{Статус проекта}{В данный момент реализован функционал необходимый для опытной эксплуатации. Программный комплекс находится на стадии тестирования заказчиком, есть данные для г.Киров (Кировская область)}
         \cvitem{Роли в проекте}{\textit{На разных стадиях проекта моя роль в нём менялась. Ниже представлены роли и обязанности в каждой из них}
         \phantomsection\label{phantomsec:gymbo-front-lead}
         \newline \textbf{\textit{Ноябрь 2020 –- настоящее время}: Ведущий Angular разработчик}}
         \cvlistitem{Личное решение задач в свободное от управления командой время}
         \cvlistitem{Управление командой фронтенд разработчиков, до 3 человек(не включая себя)}
         \cvlistitem{Планирование и контроль сроков исполнения}
         \cvlistitem{Делегирование задач конкретным исполнителям}
         \cvlistitem{Проработка и утверждение способов реализации функционала}
         \cvlistitem{Ревью кода}

         \phantomsection\label{phantomsec:gymbo-lead}
         \cvitem{}{\textbf{\textit{Сентябрь -- Ноябрь 2020}: Управление разработкой проекта}}
         \cvlistitem{Управление продуктовой командой из 8 человек: 2 бэкендера, 2 мобильных разработчика, фронтенд разработчик, тестировщик, дизайнер, ux специалист}
         \cvlistitem{Постановка задач}
         \cvlistitem{Утверждение ключевых технических решений}
         \cvlistitem{Делегирование задач конкретным исполнителям}
         \cvlistitem{Планирование и контроль сроков исполнения}

         \cvitem{}{\textbf{\textit{Январь -- Июль 2020}: Ведущий бэкенд разработчик, без непосредственного написания кода}}
         \cvlistitem{Утверждение технических решений для решения задач, которые ставил проджект менеджер}
         \cvlistitem{Ревью кода}
         \cvlistitem{Технические консультации}

         \cvitem{Личный вклад}{~
         \newline \textbf{Фронтенд}}
         \cvlistitem{Выбрал стек технологий и определил стандарты написания кода, заложил базу проекта}
         \cvlistitem{Лично реализовал большУю часть функционала}
         \cvlistitem{Управлял ходом разработки фронтенда}

         \cvitem{}{\textbf{Бэкенд}}
         \cvlistitem{Выбрал стек технологий и определил стандарты написания кода, заложил базу проекта}
         \cvlistitem{После передачи разработки бэкенда выделенной команде, довольно длительное время проводил ревью кода и участвовал в принятии архитектурных решений с правом вето}

         \cvitem{}{\textbf{Прочее}}
         \cvlistitem{Определил принципы версионирования, описания чейнджлогов, написания сommit messages, branching model}
         \cvlistitem{Разделял с проджект менеджером обязанности бизнес-аналитика, за неимением отдельного сотрудника, который бы выполнял эту роль. Полностью эта роль была с меня снята только в ноябре -- декабре 2020}
         \cvlistitem{Участвовал в выборе технологии для реализации мобильного приложения. В результате сравнения аналогов выбор был сделан в пользу Flutter}
         \cvlistitem{В сентябре 2020 из проекта ушёл проджект менеджер. На время поисков нового человека временно взял на себя его обязанности . После того как новый менеджер был найден, ввёл его в курс дела и постепенно передал управление}


         \cvitem{Использовал технологии}{
            \textbf{Фронтенд}: Angular 11, Nx monorepo, Angular Material, Angular Universal, NgRx, линтинг на основе @tinkoff/linters
            \newline \textbf{Бэкенд}: KoaJS, Typescript, Type ORM, собственная сборка ESLint + Prettier.}
         \cvitem{Адрес}{\url{https://gymbo.fitness}}
         \cvitem{}{}

      \subsection{Порталы турниров Free-Fire Pro League}\label{subsec:ffpl}
         \cvitem{Периоды участия}{Июль -- август 2020; декабрь 2020 -- январь 2021}
         \cvitem{Описание}{Информационные сайты для поддержки турниров по мобильной игре Free Fire. Помимо самих порталов, содержали панель администратора для добавления указания сыгранных айди матчей и добавления списков команд. Данные о матчах, которые добавлялись из панели администратора, загружались по API с серверов разработчиков игры, разбирались, и на основе них высчитывалась статистика по турниру и выдавалась как API для сайта, а так же для титровальной машины, используемой на трансляции}
         \cvitem{Роль в проекте}{Разработчик full-stack}
         \cvitem{Личный вклад}{Разработал сайт, бэкенд и панель администратора. На начальном этапе разработки сайта для первого сезона, была поддержка от верстальщика. Работы по второму сезону были проведены полностью сольно.}
         \cvitem{Использовал технологии}{Angular, NestJS, Type ORM, Nx monorepo, MySQL}
         \cvitem{Адреса}{Pro League Сезон 2 -- \url{https://ffpl.garena.com/FFPLS2} \newline Pro League Сезон 1 -- \url{https://ffpl.garena.com/FFPLS1}}
         \cvitem{}{}

      \subsection{Инфомат ЕМИАС}\label{subsec:infomat}
         \cvitem{Период участия}{Январь -- май 2020}
         \cvitem{Описание}{Информационные киоски для ЕМИАС. Содержат в себе функционал самозаписи на приём, электронной очереди, справочную информацию и т.п. Для пользовательского интерфейса использовался Angular 8, вёрстка происходила под 3 возможных разрешения инфоматов, запуск в конкретной версии Firefox. Проект имеет свой backend-for-frontend, который собирает информацию из сервисов ЕМИАС по SOAP и выдаёт агрегированные данные для информата в JSON. Помимо этого реализованы сервис для опроса оборудования и его мок для эмуляции работы на локальной машине}
         \cvitem{Роль в проекте}{Фронтенд разработчик (Angular) по системе outstaffing. Работа в паре с разработчиком из штатной команды. Взаимное ревью кода, обсуждение технических решений, планирование сроков. Еженедельное участие в статусах}
         \cvitem{Личный вклад}{Настроил работу связки Prettier + ESLint в проекте согласно представленному корпоративному документу с правилами оформления кода (до этого в проекте не был настроен линтинг). Благодаря этому был ряд потенциальных хрупких мест, реализация которых была скорректирована.\newline Принимал активное участив в замене старых сервисов получения данных на новую ревизию: использовалась как прямая замена на аналоги, так и разработка адаптеров, если подобрать полный аналог не удавалось. Разработал функционал для поиска поликлиники для конкретного дома, производил доработки сервисной страницы, участвовал в разработке функционала совместной записи. Так же произвёл ряд менее значительных доработок и правок }
         \cvitem{Использовал технологии}{Angular 8, в процессе работы проект переведён на Angular 9}
         \cvitem{Адрес}{Лечебные учреждения Москвы, входящие в ЕМИАС}
         \cvitem{}{}

      \subsection{Портал турнира Free-Fire Russian League}\label{subsec:ffrl}
         \cvitem{Период участия}{Декабрь 2019-- январь 2020}
         \cvitem{Описание}{Портал состоит из сайта для регистрации участников и отображения статистики турнира, панели администратора турнира, из которой происходила загрузка результатов за матч и добавление их в таблицу в google таблицу и бэкенда. Из google таблицы данные забирались титровальной машиной для вывода на трансляции матчей и сервером для вывода на сайте}
         \cvitem{Роль в проекте}{Разработчик full-stack}
         \cvitem{Личный вклад}{Разработал бэкенд и панель администратора. На основе предоставленной верстальщиком статичной вёрстки добавил интерактив: сбор заявок на участие в турнире, отображение статистики турнира}
         \cvitem{Использовал технологии}{ExpressJS, Handlebars, Angular 7, Webpack}
         \cvitem{Адрес}{Russian League Сезон 1, январь 2020 -- \url{https://ffrl.ru}}
         \cvitem{}{}

      \subsection{Система ДБО банка "Кубань кредит" для бизнеса}\label{subsec:kb-kr}
         \cvitem{Период участия}{Март 2019 -- ноябрь 2019}
         \cvitem{Описание}{Система дистанционного банковского обслуживания банка "Кубань кредит" для бизнеса на основе решений от iSimpleBank. Возрастной проект на Angular 2, при написании которого довольно скудно использовались такие принципы, как DRY и KISS}
         \cvitem{Роль в проекте}{Фронтенд разработчик (Angular) по системе outstaffing. Исполнение задач, поставленных в JIRA, взаимодействие с аналитиками заказчика для уточнения условий}
         \cvitem{Личный вклад}{Разработан функционал: конвертации валют с точной математикой(Number оказался недостаточно точен), виджет с курсами валют(с получением актуальных данных по websocket), разработан функционал для зарплатного проекта, а так же произвёл ряд менее значительных доработок}
         \cvitem{Использовал технологии}{Angular 2}
         \cvitem{Адрес}{\url{https://kb.kubankredit.ru}}
         \cvitem{}{}

      \subsection{Новостной сайт Flexer.media}\label{subsec:flexer}
         \cvitem{Период участия}{Июнь 2018 -- март 2019}
         \cvitem{Описание}{Новостной сайт с агрегацией лент сообщений из telegram. Предполагалось, что для каждого региона России будут созданы подборки политических телеграм каналов и будут выходить регулярные новости и аналитика политической жизни региона. Не дожил до релиза, был выпущен первый рабочий прототип, выполняющий основной функционал, но тема не смогла получить должное развитие. Это была попытка создать стартап с несколькими друзьями, но не хватило накопленных средств чтоб дожить до релиза, в результате тема стухла}
         \cvitem{Статус проекта}{Закрыт}
         \cvitem{Роль в проекте}{Основной разработчик платформы: фронт, бэк, админка. Второй разработчик отвечал за развёртывание и настройку сервера и обеспечение агрегации данных из телеграмма}
         \cvitem{Личный вклад}{Разработал вполне работающий MVP для фронтенда и админки и бОльшую часть бэкенда. Система в целом работала, но явно не выдержала бы большого наплыва посетителей, так как полагалась на системы реактивности от Meteor. В проекте не был заложен SSR, так что страницы не могли нормально индексироваться поисковыми системами}
         \cvitem{Использовал технологии}{Meteor, React 16.5, material-ui, MongoDB}
         \cvitem{}{}

      \subsection{Проекты Инжинирингового центра ВятГУ}\label{subsec:vyatsu}
         \cvitem{Период участия}{Август 2017 -- октябрь 2018}
         \cvitem{Описание}{Инжиниринговый центр ВятГУ в рамках своей деятельности проводил работы как на заказ, так и прорабатывал перспективные темы на основе собственной инициативы. Основная деятельность - конструирование и изготовление аппаратных и программно-аппаратных комплексов для промышленности. Помимо программно-аппаратных комплексов, велись работы и над чисто программными продуктами.}
         \cvitem{Роль}{Инженер-программист}
         \cvitem{Личный вклад}{~\newlineУчаствовал в ряде проектов разной направленности:}
         \cvlistitem{Работа над системой общения на основе Java~Spring~Boot, Vanilla~JavaScript и Rabbit~MQ}
         \cvlistitem{Участвовал в проработке системы заключения контрактов c заверением в blockchain~сети~Ethereum}
         \cvlistitem{Разработка различных скриптов на Python для автоматизации процессов. В том числе как частей программно-аппаратных комплексов}
         \cvlistitem{Доработка Delphi-программы управления газоанализатором Микрогаз-ФМ}
         \cvlistitem{Разработка и доработка существующих сайтов на Wordpress}

\end{document}
