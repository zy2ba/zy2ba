% !TeX spellcheck = russian-aot
% !TXS template
\documentclass[11pt,a4paper,sans,english,russian]{moderncv}        % possible options include font size ('10pt', '11pt' and '12pt'), paper size ('a4paper', 'letterpaper', 'a5paper', 'legalpaper', 'executivepaper' and 'landscape') and font family ('sans' and 'roman')
\moderncvstyle{classic}                             % style options are 'casual' (default), 'classic', 'oldstyle' and 'banking'
\moderncvcolor{blue}                               % color options 'blue' (default), 'orange', 'green', 'red', 'purple', 'grey' and 'black'
\moderncvicons{awesome}
%\nopagenumbers{}                                  % uncomment to suppress automatic page numbering for CVs longer than one page
\usepackage[T2A]{fontenc}
\usepackage[utf8]{inputenc}                       % if you are not using xelatex ou lualatex, replace by the encoding you are using
\usepackage{lmodern}

\usepackage[scale=0.75,a4paper]{geometry}
\usepackage{babel}

\usepackage[unicode]{hyperref}
\definecolor{linkcolour}{rgb}{0,0.2,0.6}
\hypersetup{colorlinks,breaklinks,urlcolor=linkcolour, linkcolor=linkcolour}

%----------------------------------------------------------------------------------
%            personal data
%----------------------------------------------------------------------------------
\firstname{Илья}%<first name%:columnShift:-1,persistent%>
\familyname{Зяблицев}%<family name%:columnShift:-1,persistent%>
\title{Frontend-разработчик (Angular)}%<Resumé title%:columnShift:-1,persistent%>                               % optional, remove/comment the line if not wanted
\address{г~Киров}{}{}%<street and number%:columnShift:-5,persistent%>%<postcode city%:columnShift:-3,persistent%>%<country%:columnShift:-1,persistent%>         % optional, remove/comment the line if not wanted; the "country" arguments can be omitted or provided empty
\mobile{+79123679963}%<mobile number%:columnShift:-1,persistent%>                          % optional, remove/comment the line if not wanted
%\phone{}%<phone number%:columnShift:-1,persistent%>                           % optional, remove/comment the line if not wanted
%\fax{}%<fax number%:columnShift:-1,persistent%>                             % optional, remove/comment the line if not wanted
\email{zy2bas@gmail.com}%<email%:columnShift:-1,persistent%>                               % optional, remove/comment the line if not wanted
%\homepage{}%<home page%:columnShift:-1,persistent%>                         % optional, remove/comment the line if not wanted
\extrainfo{\faPaperPlane~@zy2ba}%<additional information%:columnShift:-1,persistent%>                 % optional, remove/comment the line if not wanted
\photo[64pt][0.1pt]{photo.png}%<photo.jpg%:columnShift:-1,persistent%>                       % optional, uncomment the line if wanted; '64pt' is the height the picture must be resized to, 0.4pt is the thickness of the frame around it (put it to 0pt for no frame) and 'picture' is the name of the picture file
%\quote{}%<some quote%:columnShift:-1,persistent%>                                 % optional, remove/comment the line if not wanted
%
\renewcommand{\sfdefault}{cmr}

\usepackage{url}
\usepackage{xcolor}
\usepackage{microtype}


\def\UrlFont{\em}

\begin{document}


%-----       resume       ---------------------------------------------------------
   \makecvtitle


   \section{Образование}\label{sec:education}
      \cventry{--}{Бакалавриат, 2017~г}{Вятский государственный университет}{\newline кафедра ЭВМ}{направление "Информатика и вычислительная техника"}{}


   \section{Резюме}\label{sec:resume}
      \cvitem{--}{Опыт в Angular 2+ года}
      \cvitem{--}{Опыт в JavaScript разработке 3 года}
      \cvitem{--}{Есть опыт удалённой работы в крупных компаниях по outstaffing. Работал над информационными киосками ЕМИАС в Solit~Clouds, а так же над системой онлайн-банкинга iSimpleBank в iSimpleLab}
      \cvitem{--}{Имею опыт работы с node js фреймфорками (koa, nest, express) с использованием ORM typeorm и sequelize}
      \cvitem{--}{Есть опыт работы ведущим разработчиком по схеме "Играющий тренер". До 3 фронтенд-разработчиков под руководством. \newline В обязанности входили: ревью кода, делегирование задач, определение и контроль сроков, проработка и/или утверждение реализации функционала, помощь в решении задач. В свободное время брал задачи и решал лично}
      \cvitem{--}{Есть опыт управления командой разработчиков до 6 человек(фронтенд, бэкенд, мобильная разработка). \newline В обязанности входили: постановка задач, определение и контроль сроков, утверждение ключевых технических решений}


   \section{Опыт}\label{sec:exp}
      \cvitem{\textit{Подсказка:}}{\textit{подробности о проектах смотри в разделе \hyperref[sec:projects]{"Участие в проектах"}}}

      \subsection{Март 2019 -- настоящее время  \newline Synaptik.}\label{subsec:synaptik}

         \cvitem{\textit{Примечание:}}{\textit{С марта 2019~г работаю в компании Synaptik(\url{https://synaptik.ru}). Весь опыт в Angular получен на этом месте работы. Однако в процессе работы круг обязанностей неоднократно менялся. Потому данный опыт работы разбит на соответствующие подпункты}}

         \cvitem{Декабрь~2020 -- настоящее время}
         {
               {Ведущий Angular разработчик на проекте Gymbo.fitness (\url{https://gymbo.fitness}).}
            \newline
            Обязанности:}
         \cvlistitem{Управление командой из 2-3 человек}
         \cvlistitem{Решение задач лично}
         \cvlistitem{Делегирование задач конкретным исполнителям}
         \cvlistitem{Проработка и утверждение способов реализации функционала}
         \cvlistitem{Ревью кода}
         \cvlistitem{Определение и контроль сроков исполнения}

         \cventry{Сентябрь~-- Ноябрь~2020}{Управление разработкой проекта \hyperref[subsec:gymbo]{Gymbo, cм раздел "Справка о проектах"} (\url{https://gymbo.fitness})}{В данный момент находится на стадии тестирования заказчиком, есть данные для г. Киров (Кировская область)}{}{}{}

         \cvlistitem{ Управление продуктовой командой из 8 человек: 2 бэкендера, 2 мобильных разработчика, фронтенд разработчик, тестировщик, дизайнер, ux специалист}
         \cvlistitem{Постановка задач}
         \cvlistitem{Утверждение ключевых технических решений}
         \cvlistitem{Делегирование задач конкретным исполнителям}
         \cvlistitem{Определение и контроль сроков исполнения}

      \subsection{\textbf{Сентябрь~-- Ноябрь~2020}}
         \cvitem{Роль в проекте:}
         {
               { Фронтенд разработчик (Angular), Бэкенд разработчик (NestJS). разработчик портала для турнира по игре Free-Fire}
         }
         \cvitem{Проект:}{Портал турнира FREE FIRE PRO LEAGUE}
         \cvitem{Обязанности:}{~}
         \cvlistitem{Разработка фронтенда (https://ffpl.garena.com/FFPLS1)}
         \cvlistitem{Разработка бэкенда: получение информации о матчах по API, разбор полученных данных, предоставление результатов для фронтенда и титровальной машины}
         \cvlistitem{Разработка панели администратор получение информации о матчах по API, разбор полученных данных, предоставление результатов для фронтенда и титровальной машины}

         \cvcolumn{Обязанности}{\cvlistitem{Разработка фронтенда (https://ffpl.garena.com/FFPLS1)}
         \cvlistitem{Разработка бэкенда: получение информации о матчах по API, разбор полученных данных, предоставление результатов для фронтенда и титровальной машины}
         \cvlistitem{Разработка панели администратор получение информации о матчах по API, разбор полученных данных, предоставление результатов для фронтенда и титровальной машины}    }


   \section{Участие в проектах}\label{sec:projects}

      \subsection{Gymbo}\label{subsec:gymbo}
         \cvitem{~}{123}


\end{document}
