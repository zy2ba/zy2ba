% !TeX spellcheck = russian-aot
% !TXS template
\documentclass[11pt,a4paper,sans,english,russian]{moderncv}        % possible options include font size ('10pt', '11pt' and '12pt'), paper size ('a4paper', 'letterpaper', 'a5paper', 'legalpaper', 'executivepaper' and 'landscape') and font family ('sans' and 'roman')
\moderncvstyle{classic}                             % style options are 'casual' (default), 'classic', 'oldstyle' and 'banking'
\moderncvcolor{blue}                               % color options 'blue' (default), 'orange', 'green', 'red', 'purple', 'grey' and 'black'
\moderncvicons{awesome}
%\nopagenumbers{}                                  % uncomment to suppress automatic page numbering for CVs longer than one page
\usepackage[T2A]{fontenc}
\usepackage[utf8]{inputenc}                       % if you are not using xelatex ou lualatex, replace by the encoding you are using
\usepackage{lmodern}

\usepackage[scale=0.75,a4paper]{geometry}
\usepackage{babel}

\usepackage[unicode]{hyperref}
\definecolor{linkcolour}{rgb}{0,0.2,0.6}
\hypersetup{colorlinks,breaklinks,urlcolor=linkcolour, linkcolor=linkcolour}

%----------------------------------------------------------------------------------
%            personal data
%----------------------------------------------------------------------------------
\firstname{Илья}%<first name%:columnShift:-1,persistent%>
\familyname{Зяблицев}%<family name%:columnShift:-1,persistent%>
\title{Frontend-разработчик (Angular)}%<Resumé title%:columnShift:-1,persistent%>                               % optional, remove/comment the line if not wanted
\address{г~Киров}{}{}%<street and number%:columnShift:-5,persistent%>%<postcode city%:columnShift:-3,persistent%>%<country%:columnShift:-1,persistent%>         % optional, remove/comment the line if not wanted; the "country" arguments can be omitted or provided empty
\mobile{+79123679963}%<mobile number%:columnShift:-1,persistent%>                          % optional, remove/comment the line if not wanted
%\phone{}%<phone number%:columnShift:-1,persistent%>                           % optional, remove/comment the line if not wanted
%\fax{}%<fax number%:columnShift:-1,persistent%>                             % optional, remove/comment the line if not wanted
\email{zy2bas@gmail.com}%<email%:columnShift:-1,persistent%>                               % optional, remove/comment the line if not wanted
%\homepage{}%<home page%:columnShift:-1,persistent%>                         % optional, remove/comment the line if not wanted
\extrainfo{\faPaperPlane~@zy2ba}%<additional information%:columnShift:-1,persistent%>                 % optional, remove/comment the line if not wanted
\photo[64pt][0.1pt]{photo.png}%<photo.jpg%:columnShift:-1,persistent%>                       % optional, uncomment the line if wanted; '64pt' is the height the picture must be resized to, 0.4pt is the thickness of the frame around it (put it to 0pt for no frame) and 'picture' is the name of the picture file
%\quote{}%<some quote%:columnShift:-1,persistent%>                                 % optional, remove/comment the line if not wanted
%
\renewcommand{\sfdefault}{cmr}

\usepackage{url}
\usepackage{xcolor}
\usepackage{microtype}

\def\UrlFont{\em}

\begin{document}


%-----       resume       ---------------------------------------------------------
   \makecvtitle


   \section{Образование}\label{sec:education}
      \cventry{--}{Бакалавриат, 2017~г}{Вятский государственный университет}{\newline кафедра ЭВМ}{направление "Информатика и вычислительная техника"}{}


   \section{Резюме}\label{sec:resume}
      \cvitem{--}{Опыт в Angular 2+ года}
      \cvitem{--}{Опыт в JavaScript разработке 3 года}
      \cvitem{--}{Есть опыт работы с Nx monorepo, NgRx, Polymorpheus, Angular Universal c избирательным Transfer State}
      \cvitem{--}{Есть опыт удалённой работы в крупных компаниях по outstaffing. Работал над \hyperref[subsec:solit-clouds-outstaff]{информационными киосками ЕМИАС в Solit~Clouds}, а так же над \hyperref[subsec:i-simple-outstuff]{системой онлайн-банкинга iSimpleBank в iSimpleLab}}
      \cvitem{--}{Имею опыт работы с node js фреймворками (koa, nest, express) с использованием ORM typeorm и sequelize}
      \cvitem{--}{\hyperref[subsec:gymbo-front-lead]{Есть опыт работы ведущим разработчиком по схеме "Играющий тренер".} До 3 фронтенд-разработчиков под руководством. \newline В обязанности входили: ревью кода, делегирование задач, определение и контроль сроков, проработка и/или утверждение реализации функционала, помощь в решении задач. В свободное время брал задачи и решал лично}
      \cvitem{--}{\hyperref[subsec:gymbo-lead]{Есть опыт управления продуктовой командой из 8 человек.} \newline В обязанности входили: постановка задач, определение и контроль сроков, утверждение ключевых технических решений}


   \section{Опыт}\label{sec:exp}
      \cvitem{\textit{Подсказка:}}{\textit{подробности о проектах смотри в разделе \hyperref[sec:projects]{"Участие в проектах"}}}

      \subsection{Март 2019 -- настоящее время  \newline Synaptik.}\label{subsec:synaptik}

         \cvitem{\textit{Примечание:}}{\textit{С марта 2019~г работаю в компании Synaptik(\url{https://synaptik.ru}). Весь опыт в Angular получен на этом месте работы. Однако в процессе работы круг обязанностей неоднократно менялся. Потому данный опыт работы разбит на соответствующие подпункты}}

      \subsection{}\label{subsec:gymbo-front-lead}
         \cventry{\textbf{Декабрь 2020 -- настоящее время}}{Ведущий Angular разработчик на проекте \hyperref[subsec:gymbo]{Gymbo}}{}{}{}{}
         \cvitem{}{Обязанности:}
         \cvlistitem{Личное решение задач в свободное от управления командой время}
         \cvlistitem{Управление командой из 2--3 человек(не включая себя)}
         \cvlistitem{Определение и контроль сроков исполнения}
         \cvlistitem{Делегирование задач конкретным исполнителям}
         \cvlistitem{Проработка и утверждение способов реализации функционала}
         \cvlistitem{Ревью кода}

      \subsection{}\label{subsec:gymbo-lead}
         \cventry{\textbf{Сентябрь -- Ноябрь 2020}}{Управление разработкой проекта \hyperref[subsec:gymbo]{Gymbo}}{}{}{}{}
         \cvlistitem{Управление продуктовой командой из 8 человек: 2 бэкендера, 2 мобильных разработчика, фронтенд разработчик, тестировщик, дизайнер, ux специалист}
         \cvlistitem{Постановка задач}
         \cvlistitem{Утверждение ключевых технических решений}
         \cvlistitem{Делегирование задач конкретным исполнителям}
         \cvlistitem{Определение и контроль сроков исполнения}

      \subsection{}\label{subsec:ffpl-s2}
         \cventry{\textbf{Сентябрь -- Ноябрь 2020}}{Портал турнира Free Fire Pro League Season 2}{}{}{}{}
         \cvitem{Роль в проекте:}
         {
            Фронтенд разработчик (Angular), Бэкенд разработчик (NestJS). разработчик портала для \href{subsec:free-fire}{турнира по игре Free-Fire}
         }
         \cvitem{Проект:}{Портал турнира FREE FIRE PRO LEAGUE}
         \cvitem{Обязанности:}{\cvlistitem{Разработка фронтенда (https://ffpl.garena.com/FFPLS1)}
         }
         \cvlistitem{Разработка бэкенда: получение информации о матчах по API, разбор полученных данных, предоставление результатов для фронтенда и титровальной машины}
         \cvlistitem{Разработка панели администратор получение информации о матчах по API, разбор полученных данных, предоставление результатов для фронтенда и титровальной машины}

         \cvcolumn{Обязанности}{\cvlistitem{Разработка фронтенда (https://ffpl.garena.com/FFPLS1)}
         \cvlistitem{Разработка бэкенда: получение информации о матчах по API, разбор полученных данных, предоставление результатов для фронтенда и титровальной машины}
         \cvlistitem{Разработка панели администратор получение информации о матчах по API, разбор полученных данных, предоставление результатов для фронтенда и титровальной машины}    }

      \subsection{}\label{subsec:solit-clouds-outstaff}
         \cventry{\textbf{Январь~-- Май~2020}}{Разработка функционала для инфоматов ЕМИАС}{}{}{}{}
         \cvitem{Роль в проекте}
         {
            Фронтенд разработчик (Angular) по системе outstaffing.
         }

      \subsection{}\label{subsec:i-simple-outstuff}
         \cventry{Март~-- Ноябрь~2019}{Разработка функционала для систем онлайн банкинга iSimpleBank}{}{}{}{}
         \cvitem{Роль в проекте}{Фронтенд разработчик (Angular) по системе outstaffing.}
         \cvitem{Личный вклад}{Разработка}


   \section{Участие в проектах}\label{sec:projects}

      \subsection{Gymbo}\label{subsec:gymbo}
         \cvitem{Период участия}{Январь 2020 -- настоящее время}
         \cvitem{Описание}{Платформа с каталогом фитнес-клубов с возможностью покупки абонементов в зал и персональных тренировок. Программный комплекс включает сайт, мобильное приложение, бэкенд и панель администратора сервиса}
         \cvitem{Статус проекта}{В данный момент релизный функционал находится на стадии тестирования заказчиком, есть данные для г.Киров (Кировская область)}
         \cvitem{Роли в проекте}{\textit{На разных стадиях проекта моя роль в нём менялась. Ниже представлены роли и обязанности в каждой из них}
         \newline \textbf{\textit{Ноябрь 2020 –- настоящее время}: Ведущий Angular разработчик}}
         \cvlistitem{Личное решение задач в свободное от управления командой время}
         \cvlistitem{Управление командой из 2--3 человек(не включая себя)}
         \cvlistitem{Определение и контроль сроков исполнения}
         \cvlistitem{Делегирование задач конкретным исполнителям}
         \cvlistitem{Проработка и утверждение способов реализации функционала}
         \cvlistitem{Ревью кода}

         \cvitem{}{\textbf{\textit{Сентябрь -- Ноябрь 2020}: Управление разработкой проекта}}
         \cvlistitem{Управление продуктовой командой из 8 человек: 2 бэкендера, 2 мобильных разработчика, фронтенд разработчик, тестировщик, дизайнер, ux специалист}
         \cvlistitem{Постановка задач}
         \cvlistitem{Утверждение ключевых технических решений}
         \cvlistitem{Делегирование задач конкретным исполнителям}
         \cvlistitem{Определение и контроль сроков исполнения}

         \cvitem{}{\textbf{\textit{Январь -- Июль 2020}: Ведущий бэкенд разработчик, без непосредственного написания кода}}
         \cvlistitem{Выбор стека технологий и определение стандартов написания кода}
         \cvlistitem{Утверждение технических решений для решения задач, которые ставил Project Manager}
         \cvlistitem{Ревью кода}
         \cvlistitem{Технические консультации}

         \cvitem{Личный вклад}{~
         \newline \textbf{Фронтенд}}
         \cvlistitem{Выбрал стек технологий и определил стандарты написания кода, заложил базу проекта}
         \cvlistitem{Лично реализовал большУю часть функционала}
         \cvlistitem{Управлял ходом разработки фронтенда}

         \cvitem{}{\textbf{Бэкенд}}
         \cvlistitem{Выбрал стек технологий и определил стандарты написания кода, заложил базу проекта}
         \cvlistitem{После передачи разработки бэкенда отдельным разработчикам, довольно длительное время проводил ревью кода, а так же участвовал в принятии архитектурных решений с правом вето}

         \cvitem{}{\textbf{Прочее}}
         \cvlistitem{Разделял с проджект менеджером обязанности бизнес-аналитика, за неимением отдельного сотрудника, который бы выполнял эту роль. Полностью эта роль была с меня снята только в ноябре -- декабре 2020}
         \cvlistitem{Участвовал в выборе технологии для реализации мобильного приложения. В результате сравнения аналогов выбор был сделан в пользу Flutter}


         \cvitem{Использовал технологии}{
            \textbf{Фронтенд}: Angular 11, Nx monorepo, Angular Material, Angular Universal, NgRx, линтинг на основе @tinkoff/linters
            \newline \textbf{Бэкенд}: KoaJS, Typescript, Type ORM, собственная сборка ESLint+Prettier.}
         \cvitem{Адрес}{\url{https://gymbo.fitness}}
         \cvitem{}{}

      \subsection{Порталы турниров Free-Fire Pro League}\label{subsec:ffpl}
         \cvitem{Периоды участия}{Июль -- август 2020; декабрь 2020 -- январь 2021}
         \cvitem{Описание}{Информационные сайты для поддержки турниров по мобильной игре Free Fire. Помимо самих порталов, содержали панель администратора для добавления указания сыгранных айди матчей и добавления списков команд. Данные о матчах, которые добавлялись из панели администратора, загружались по API с серверов разработчиков игры, разбирались, и на основе них высчитывалась статистика по турниру и выдавалась как АПИ для сайта, а так же для титровальной машины, используемой на трансляции}
         \cvitem{Роль в проекте}{Разработчик full-stack}
         \cvitem{Личный вклад}{Разработал непосредственно сайты, бэкенд и панель администратора. На первичном этапе первого сезона была поддержка от верстальщика, в дальнейшем сольно доводил сайт первого сезона, а затем на основе имеющихся материалов дорабатывал сайт под второй сезон турнира}
         \cvitem{Использовал технологии}{Angular, NestJS, Type ORM, Nx monorepo, MySQL}
         \cvitem{Адреса}{Pro League Сезон 2 -- \url{https://ffpl.garena.com/FFPLS2}, Pro League Сезон 1 -- \url{https://ffpl.garena.com/FFPLS1}}
         \cvitem{}{}

      \subsection{Инфомат ЕМИАС}\label{subsec:infomat}
         \cvitem{Период участия}{Январь -- май 2020}
         \cvitem{Описание}{Информационные киоски для ЕМИАС. Содержат в себе функционал самозаписи на приём, электронной очереди, справочную информацию и т.п. Для пользовательского интерфейса использовался Angular 8, вёрстка происходила под 3 возможных разрешения инфоматов, запуск в конкретной версии Firefox. Проект имеет свой backend-for-frontend, который собирает информацию из сервисов ЕМИАС по SOAP и выдаёт агрегированные данные для информата в JSON. Так же на Python написаны сервис для опроса оборудования, а так же его мок для эмуляции работы на локальной машине}
         \cvitem{Роль в проекте}{Фронтенд разработчик (Angular) по системе outstaffing. Работа в паре с разработчиком из штатной команды. Взаимное ревью кода, обсуждение технических решений, оценка сроков. Еженедельное участие в статусах}
         \cvitem{Личный вклад}{Настроил работу связки Prettier + ESLint в проекте согласно представленному корпоративному документу с правилами оформления кода (до этого в проекте не был настроен линтинг). Благодаря этому был ряд потенциальных хрупких мест, реализация которых была скорректирована.\newline Принимал активное участив в замене старых сервисов получения данных на новую ревизию: использовалась как прямая замена на аналоги, так и разработка адаптеров, если подобрать полный аналог не удавалось. Разработал функционал для поиска поликлиники для конкретного дома, производил доработки сервисной страницы, участвовал в разработке функционала совместной записи. Так же произвёл ряд менее значительных доработок и правок }
         \cvitem{Использовал технологии}{Angular 8, в процессе работы проект переведён на Angular 9}
         \cvitem{Адрес}{Лечебные учреждения Москвы, входящие в ЕМИАС}
         \cvitem{}{}

      \subsection{Портал турнира Free-Fire Russian League}\label{subsec:ffrl}
         \cvitem{Период участия}{Декабрь 2019-- январь 2020}
         \cvitem{Описание}{Сайт для регистрации участников и отображения статистики турнира. Так же разработана панель администратора, из которой происходила загрузка результатов за матч и добавление их в таблицу в google таблицу. Из google таблицы данные забирались для титровальной машины, а так же для вывода на сайте}
         \cvitem{Роль в проекте}{Разработчик full-stack}
         \cvitem{Личный вклад}{Разработал бэкенд и панель администратора, а так же на основе предоставленной вёрстки добавил интерактив: сбор заявок на участие в турнире, а так же отображение статистики турнира}
         \cvitem{Использовал технологии}{ExpressJS + Handlebars + Angular 7 + Webpack}
         \cvitem{Адрес}{Russian League Сезон 1, январь 2020 -- \url{https://ffrl.ru}}
         \cvitem{}{}

      \subsection{Система ДБО банка "Кубань кредит" для бизнеса}\label{subsec:kb-kr}
         \cvitem{Период участия}{Март 2019 -- ноябрь 2019}
         \cvitem{Описание}{Система дистанционного банковского обслуживания банка "Кубань кредит" для бизнеса на основе решений от iSimpleBank. Возрастной проект времён зари Angular 2, при написании которого довольно скудно использовались такие подходы, как DRY и KISS}
         \cvitem{Роль в проекте}{Фронтенд разработчик (Angular) по системе outstaffing. Исполнение задач, поставленных в JIRA, взаимодействие с аналитиками заказчика для уточнения условий}
         \cvitem{Личный вклад}{Разработан функционал: конвертации валют с точной математикой(Number оказался недостаточно точен), виджет с курсами валют(с получением актуальных данных по websocket), разработан функционал для зарплатного проекта, а так же произвёл ряд менее значительных доработок}
         \cvitem{Использовал технологии}{Angular 2}
         \cvitem{Адрес}{\url{https://kb.kubankredit.ru}}
         \cvitem{}{}

      \subsection{Новостной сайт Flexer.media}\label{subsec:flexer}
         \cvitem{Период участия}{Июнь 2018 -- март 2019}
         \cvitem{Описание}{Новостной сайт с агрегацией лент сообщений из telegram. Предполагалось, что для каждого региона России будут созданы подборки политических телеграм каналов, а так же будут регулярные новости и аналитика политической жизни региона. Не дожил до релиза, был выпущен первый рабочий прототип, выполняющий основной функционал, но тема не смогла получить должное развитие. Это была попытка создать стартап с несколькими друзьями, но не хватило накопленных средств чтоб дожить до релиза, в результате тема стухла}
         \cvitem{Статус проекта}{Закрыт}
         \cvitem{Роль в проекте}{Основной разработчик платформы: фронт, бэк, админка. Второй разработчик отвечал за развёртывание и настройку сервера, а так же за обеспечение агрегации данных из телеграмма}
         \cvitem{Личный вклад}{Разработал вполне работающий MVP для фронтенда и админки, а так же бОльшую часть бэкенда. Система в целом работала, но явно не выдержала бы большого наплыва посетителей, так как полагалась на системы реактивности от Meteor. Так же отсутствовал SSR, так что страницы не могли нормально индексироваться поисковыми системами}
         \cvitem{Использовал технологии}{Meteor + React 16.5 + material-ui, MongoDB}
         \cvitem{}{}

      \subsection{Проекты Инжинирингового центра ВятГУ}\label{subsec:vyatsu}
         \cvitem{Период участия}{Август 2017 -- октябрь 2018}
         \cvitem{Описание}{Исполнял обязанности инженера программиста в структуре ВятГУ, именуемой "Инжиниринговый центр". Было решено открыть отдел программистов, но не было придумано зачем именно. За год с небольшим было пять попыток довести до MVP различные проекты. Однако не завершив один проект, фокус переключался на другой, да и не было чёткого понимания какой продукт должен получиться и кому он нужен. Однако в технологиях ни кто не ограничивал, потому удалось поработать с Java(Spring) + Vanilla JavaScript, Python, PHP. Так же пришлось поработать с легаси кодом на Delphi для доработки системы взаимодействия со специализированным устройством (Газоанализатор Микрогаз-ФМ)}
\end{document}
